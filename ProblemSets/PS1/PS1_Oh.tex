\documentclass{article}

% Language setting
% Replace `english' with e.g. `spanish' to change the document language
\usepackage[english]{babel}

% Set page size and margins
% Replace `letterpaper' with `a4paper' for UK/EU standard size
\usepackage[letterpaper,top=2cm,bottom=2cm,left=3cm,right=3cm,marginparwidth=1.75cm]{geometry}

% Useful packages
\usepackage{amsmath}
\usepackage{graphicx}
\usepackage[colorlinks=true, allcolors=blue]{hyperref}

\title{ECON5253 Problem Set 1}
\author{JaeSeok Oh}

\begin{document}
\maketitle


\section*{Question 5}

\begin{enumerate}
    \item Interests in Economics
    \begin{itemize}
        \item My research interest lies in the field of Industrial Organization. Particularly, I am eager to look at `Sneakers' market and firms such as Nike, New Balance, Adidas, etc.. This market has a unique characteristic for some `edition series' that firms either make a draw or sell products to the first-comers by producing much less than the quantity consumers want to buy. As a result, those who fail to buy products are willing to join the secondary market where resellers are in. Of course, the prices should be higher than the original prices, called price premium.
        \item Therefore, I would divide this whole market as two markets: Primary market and Secondary market. For the primary market, I want to look at the behaviors of sneakers' firms such as the timing of release, prices, etc.. And the matter of the secondary market is how the price has changed since the secondary market operated. 
    \end{itemize}
    \item What I want to learn in this class?
    \begin{itemize}
        \item The analysis of the secondary market is based on a `Sequential Auction' because the auction of each sneakers keeps doing for a long time period. Through `kaggle'(\url{https://www.kaggle.com}), I was able to get a data set for 50 different sneakers with 18 months time periods. It consists of almost 100,000 observations transacted at `StockX' which is well-known platform. After setting up some basic fixed effects model with different versions, I found that I need more variables. In order to get other variables, I need to do scrapping some data in the platform.
        \item Therefore, I want to learn how to scrap data on the website. This is a very specific desire, but I am also coming this class to oversee the ways to handle data and get the big picture of computer tools using in these days.
    \end{itemize}
\end{enumerate}

\section*{Question 6. Equation}
\begin{equation}
    a^{2} + b^{2} = c^{2}
\end{equation}



\bibliographystyle{alpha}
\bibliography{sample}

\end{document}
