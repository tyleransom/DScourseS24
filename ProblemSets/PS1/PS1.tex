% \documentclass{article}
\documentclass[10pt]{article}
\pagenumbering{gobble}
\usepackage[colorlinks=true, allcolors=blue]{hyperref}
\usepackage{setspace}
% Language setting
% Replace `english' with e.g. `spanish' to change the document language
\usepackage[english]{babel}

% Set page size and margins
% Replace `letterpaper' with `a4paper' for UK/EU standard size
\usepackage[letterpaper,top=0.9cm,bottom=1.5cm,left=2cm,right=2cm,marginparwidth=1.75cm]{geometry}

% Useful packages
\usepackage{amsmath}
\usepackage{graphicx}
\usepackage[colorlinks=true, allcolors=blue]{hyperref}

\title{Personal Statement: Economics and Data Science}
\author{Olubayode Ebenezer}
\date{January 24 2024}

\begin{document}
\maketitle


\section{Interest in Economics and Data Science}

My interest in economics and data science is fueled by a passion for sports economics and analytics. The core of my enthusiasm lies in the power of data to unveil key insights and enhance performance in the sports sector. Having a lifelong affinity for sports, I am captivated by the role of quantitative analysis in shaping athletic performance and strategic decisions. This interest is not just limited to the physical aspects of sports but extends to encompass strategic, health, and business considerations.

\section{ Motivation for Taking This Class}

My motivation for enrolling in this class stems from a profound interest in marrying the realms of sports with economics and data science. I am fascinated by how data-driven approaches can revolutionize sports, transforming every aspect from performance optimization to strategic business decisions. The class presents a unique opportunity to deepen my understanding of sports analytics, particularly in employing AI for biomechanical analysis. Engaging in this coursework aligns seamlessly with my goal of becoming a pivotal contributor to the sports analytics field, focusing on advanced data interpretation and application in sports economics

\subsection{Project Idea}

My projects will be focusing  on using a groundbreaking tool that utilizes AI algorithms to interchange biomechanics animations at the pitch level. This system allows for real-time monitoring, analysis, and performance enhancement for pitchers. The tool meticulously tracks key metrics such as arm angle, shoulder rotation, and kinetic sequences, providing instant alerts when a pitcher deviates from established normative ranges. This not only improves performance but also safeguards the well-being of players. Additionally, it will also generate the player datafile generator, a comprehensive analytical tool that compiles a player's biomechanical data over specified periods. This feature generates detailed reports in various formats, including CSV, PDF, and Excel, offering invaluable insights for players and coaches alike. This tool's capability to analyze and improve performance while ensuring athlete well-being showcases the intersection of data science and sports economics.

\subsection{Goals For The Class}
Here are my goals for this clas:

i. I want to acquire advanced skills in AI and machine learning, particularly in sports analytics.

ii. I want to develop a comprehensive understanding of economic impacts and models in sports business.

iii. I want to gain proficiency in predictive analytics and performance monitoring in sports.

iv. Enhance my ability to analyze and interpret complex data sets for strategic decision-making in sports management.

\subsection{Post-Graduation Plan}
Upon graduation, I aspire to become a renowned sports analyst, with a dual specialization in sports data analysis and sports marketing management. My career path will focus on democratizing sports data for broader application, analyzing sports business dynamics, and comprehensively understanding biometric and physiological data. I aim to leverage these skills to improve athletic performance and health, highlight Nigerian athletes on international platforms, and revolutionize the Nigerian sports industry into a more data-centric and strategically sophisticated domain.


\subsection{Equation}
I use LaTex  to write a Pythagorean theorem \(a^2 + b^2 = c^2\)  

\end{document}